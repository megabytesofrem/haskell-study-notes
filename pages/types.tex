% Types
\section{Types}
\subsection{Basic types}
Haskell has many primitive types such as strings, characters, integers and floating point numbers.

\begin{lstlisting}
"hello world" -- string
1234          -- integer
3.14          -- float
\end{lstlisting}

\subsection{Typeclasses}
Typeclasses are a way of sharing specific functionality between types. We can either implement our
own instances of typeclasses, or let haskell \emph{derive} them automatically.

They are kind of like Java interfaces.

Num is the generic base typeclass that all numbers derive from.
\begin{enumerate}
    \item Integral numbers (\emph{Integral} typeclass)
    \begin{itemize}
        \item \emph{Int}: fixed precision integer with a min and maximum size
        \item \emph{Integer}: supports \textbf{very} large integers
    \end{itemize}
    \item Floating point numbers (\emph{Fractional} typeclass)
    \begin {itemize}
        \item \emph{Float}: single precision floating point number
        \item \emph{Double}: double precision floating point number
    \end{itemize}
\end{enumerate}

\subsection{More on typeclasses}
Anything that derives from
\begin{itemize}
    \item \emph{Show} can be printed
    \item \emph{Read} can be read as a value
    \item \emph{Eq} can be compared for equality with \emph{==} and \emph{/=}
    \item \emph{Ord} can be compared and ordered with \emph{<} and \emph{>}
\end{itemize}

\newpage
\begin{lstlisting}
{-# LANGUAGE DuplicateRecordFields #-}

data Worker = Worker { name :: String, job :: String } deriving (Show)
data Student = Student { name :: String, school :: String } deriving (Show)

class Person a where
    getName :: a -> String
    getOccupation :: a -> String

instance Person Worker where
    getName x = name (x :: Worker)
    getOccupation x = "Working on " ++ job x

instance Person Student where
    getName x = name (x :: Student)
    getOccupation x = "Studying at " ++ school x
\end{lstlisting}
\figurename{ 1: Demonstration of a custom typeclass for Person}