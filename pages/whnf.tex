\section{Weak head normal form}
Values in Haskell are reduced to Weak head normal form (WHNF).
Weak head normal form is a type of normal form which contains both the possibility the expression has been fully evaluated/reduced (normal form) as well as the possibility that the expression has been evaluated to the point of arriving at a data constructor/lambda waiting for an argument.

These expressions \textbf{are} in WHNF:

\begin{lstlisting}[linewidth=15cm]
-- https://stackoverflow.com/questions/6872898/
(1, 1 + 1)          -- outermost part is the data constructor (,)
\x -> 2 + 2         -- outermost part is a lambda abstraction
'h' : ("e" + "llo") -- outermost part is the data constructor (:)
\end{lstlisting}

These examples \textbf{are not} in WHNF
\begin{lstlisting}[linewidth=15cm]
1 + 2               -- the outermost part here is an application of (+)
(\x -> x + 1) 2     -- the outermost part is an application of (\x -> x + 1)
"he" ++ "llo"       -- the outermost part is an application of (++)    
\end{lstlisting}

We can use \texttt{:sprint} in GHCI to print out the representation of the value in memory. Due to laziness, polymorphic types are unevaluated (thunked) until we use them. This is marked with an underscore (\_).

\begin{verbatim}
Prelude> let nums = [1..5]

Prelude> :sprint nums
nums = _

Prelude> take 2 nums
Prelude> :sprint nums
nums = 1 : 2 : _
\end{verbatim}

\footnote{The examples on WHNF were taken directly from the amazing Haskell Book}