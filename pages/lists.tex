\section{Lists and Tuples}

\subsection{Lists}
Lists have a \emph{head} and a \emph{tail}. Because of Haskell being lazily evaluated, lists can be infinite.
\emph{take} takes \emph{n} items from a list, and \emph{drop} drops \emph{n} items from a list.\\
\\
list = [1, 2, 3, 4, 5]\\
In the above list, the \emph{head} is \underline{1}, and the \emph{tail} is \underline{[2, 3, 4, 5]}

\begin{lstlisting}
list = [1, 2, 3, 4, 5]
head list -- 1
tail list -- [2, 3, 4, 5]

-- take 10 elements from an infinite list
take 10 [1..]

-- if you try to just get the entire list, it will hang GHCI
-- since it will never end because the list is infinite in size
[1..]
\end{lstlisting}

Lists can be indexed using the \emph{!!} operator. 
\begin{lstlisting}
firstItem = list !! 0 -- lists start from 0    
\end{lstlisting}

\subsection{Tuples}
Haskell has tuples, triples, and \emph{n-tuples}. Tuples have a \emph{fst} and \emph{snd} function which
respectively gets either the first \underline{or} second item. 

\emph{swap} (defined in \textbf{Data.Tuple}) swaps the items in a \emph{tuple}

\begin{lstlisting}
("char", 20)

-- a triple
("char", 20, "turtles")
\end{lstlisting}