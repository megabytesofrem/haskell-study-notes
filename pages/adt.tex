\section{Algebraic Data Types}

\subsection{How Datatypes are constructed}

\begin{verbatim}
  data Bool = False  |  True
-- [1] [2] [3] [4]  [5]  [6]
\end{verbatim}

In the above example for a \emph{Bool} datatype there are the parts broken down:
\begin{enumerate}
\item The keyword \emph{data} which signals that what follows is defining a datatype
\item Type constructor (with no arguments)
\item Equals sign which seperates the \emph{type constructor} from the \emph{data constructor}
\item Data constructor. A data constructor that takes \emph{zero} arguments is called a \textbf{nullary constructor}.
\item The pipe which denotes a \textbf{sum} type, which indicates a logical disjunction (or).
\item Constructor for the value \texttt{True} -- another \textbf{nullary constructor}.
\end{enumerate}

Let's look at another example taken from the Haskell Book.
We will look at the \emph{data constructor} and \emph{type constructors} for the list type.

\begin{verbatim}
data [] a = [ ]  | a  :  [a]
--   [ 1 ]  [2]      [3]
\end{verbatim}

\begin{enumerate}
\item Type constructor with an argument. The argument here is a polymorphic type variable,
  so the lists argument can be of different types.
\item Data constructor for the empty list.
\item Data constructor that takes two arguments: an \emph{a} and also a list of \emph{a} aka \emph{[a]}.
\end{enumerate}

\subsubsection {Kinds}
If we look closer at the list data type, we can tell based on the data type that it must be applied to a
concrete type, like \emph{Integer} or \emph{String} before we have a list of that thing.

Kinds are the types of types, or types one level up. They are represented with *. When something is a fully applied,
concrete type its kind is *. When it is \texttt{* -> *} it is waiting to be applied.

We can query the kind signature of a type constructor in GHC using \texttt{:kind} or \texttt{:k}.
\begin{verbatim}
Prelude> :k Bool
Bool :: *

Prelude> :k [Int]
[Int] :: *

Prelude> :k []
[] :: * -> *
\end{verbatim}

Both \texttt{Bool} and \texttt{[Int]} are fully applied, concrete types so their kind has no arrows. However, the
kind of \texttt{[]} is \texttt{* -> *} since it still needs to be applied to a concrete type before it itself is
a concrete type.

\subsubsection{Nullary, unary and product types}
A type can be thought of as an enumeration of constructions that have zero or \emph{more} arguments.

All of the following are valid data declarations:
\begin{lstlisting}
  -- nullary
  data Example0 =
    Example0 deriving (Eq, Show)

  -- unary
  data Example1 =
    Example1 Int deriving (Eq, Show)

  -- product of Int and String
  data Example2 =
    Example2 Int String deriving (Eq, Show)
\end{lstlisting}


\subsection{What makes them algebraic}
Algebraic datatypes (ADTs) in Haskell are algebraic because we can describe the patterns of their argumement
structures using two basic operations: sum and product. The most direct way to explain why they’re called sum
and product is to demonstrate sum and product in terms of cardinality.

The cardinality of a datatype is the number of possible values it defines. It can be as small as zero, or large as
infinite.

The cardinality of \texttt{Bool} is 2 since there are two possible values (True or False), and the cardinality of
\texttt{Int8} is 256 since it can range from -128 to 127 so 128 + 127 + 1 = 256.

\subsection{Determining cardinality}

\subsubsection{Nullary data constructors}
\begin{lstlisting}
data Example = MakeExample deriving Show
\end{lstlisting}

Since \emph{MakeExample} takes \underline{no} type arguments it's a nullary constructor.
Nullary data constructors are constants that only represent themselves as values -- so they have a cardinality of 1.

Since \texttt{MakeExample} is a single nullary value for the type \texttt{Example}, the cardinality of the type is 1.

\subsubsection{Unary constructors}
A unary data constructors takes exactly one argument. Instead of the data constructor being a constant, known value
like in nullary constructors -- the value will be constructed at run time from the argument we applied it to.

Datatypes that only contain a unary constructor \textbf{always} have the same cardinality as the type they contain.

\begin{lstlisting}
data Goats = Goats Int deriving (Eq, Show)
\end{lstlisting}

Anything that is a valid Int must \textbf{also} be a \underline{valid} argument to the \texttt{Goats} constructor.
For cardinality, this means unary constructors are the \emph{identity} function.

\subsubsection{Sum types}
To determine the cardinality of sum types, we add the cardinalities of their data constructors.
\texttt{True} and \texttt{False} take no type arguments and are nullary constructors, each with a cardinality of 1.

Now we do some arithmetic. Nullary constructors are 1, and sum types are addition of + when we are talking
about the cardinality.

\begin{verbatim}
-- how many values inhabit bool?
data Bool = False | True

True | False = C

-- given that |, the sum type syntax is addition
True + False = C

-- and that False and True are both == 1
1 + 1 = C
1 + 2 = 2

-- the cardinality is 2
C = 2
\end{verbatim}

\subsubsection{Product types}
A product types cardinality is the \textbf{product} of the cardinalities of its inhabitants.
A product type is like a struct.

\begin{lstlisting}
data QuantomBool = QuantomTrue
                 | QuantomFalse
                 | QuantomBoth deriving (Eq, Show)

data TwoQs =
  MkTwoQs QuantomBool QuantomBool
  deriving (Eq, Show)
\end{lstlisting}

The datatype \texttt{TwoQs} has one \underline{data constructor}, \texttt{MkTwoQs} that takes
two arguments -- making it a product of the two types that inhabit it.

Each argument is a \texttt{QuantomBool} which has a cardinality of 3.
Product types can also be defined in \textbf{record} syntax.

\begin{lstlisting}
-- using a data constructor
data Person = MkPerson String Int deriving (Eq, Show)

-- using record syntax
data Person = Person { name :: String
                     , age  :: Int 
                     } deriving (Eq, Show)

-- both of these are identical but the record syntax
-- generates name and age functions
\end{lstlisting}

Product types are distributive over sum types.
\begin{verbatim}
a * (b + c) -> (a * b) + (a * c)
\end{verbatim}

% example of product types being distributive
\lstinputlisting{pages/example/product-types.hs}

\subsubsection{newtype}

We use the \texttt{newtype} keyword to mark a type that can only ever have a single unary data constructor.
A \texttt{newtype} cannot be a product type, sum type or contain any nullary constructors.

However, it has some advantages of a vanilla \texttt{data} declaration. It has no runtime overhead as
it reuses the representation of the type it contains (which is allowed since it cannot be a record (\emph{product type})
or a tagged union (\emph{sum type}).

\begin{lstlisting}
newtype Goats = 
  Goats Int deriving (Eq, Show)

newtype Cows =
  Cows Int deriving (Eq, Show)

-- Now we can't accidentally mix up the Goats and Cows
tooManyGoats :: Goats -> Bool
tooManyGoats (Goats n) = n > 42
\end{lstlisting}

A \texttt{newtype} is similar to a type alias in that the distinction between them is gone by compile time.
So a \emph{String} really is a \emph{Char[]} and \emph{Goats} above is really an \emph{Int}.

However, one key difference bwteen a \texttt{newtype} and a type alias is that you can define typeclass
instances for \texttt{newtypes} that \underline{differ} from the instances for their underlying type.

\begin{lstlisting}
class TooMany a where
  tooMany :: a -> Bool

instance TooMany Int where
  tooMany n = n > 42
\end{lstlisting}

We can use it in the REPL but \underline{only} if we assign the type \texttt{Int} to whatever
number we pass in since numeric literals are polymorphic.

So this works (if we remove the \texttt{(42 :: Int)} it will blow up since we \underline{only} defined the instance
for \texttt{Int}).
\begin{verbatim}
tooMany (42 :: Int)
\end{verbatim}

Under the hood, \texttt{Goats} is still \texttt{Int} \textbf{but} the \emph{newtype} declaration
will allow us to define a custom instance.


\begin{lstlisting}
newtype Goats = Goats Int deriving Show

instance TooMany Goats where
  tooMany (Goats n) = n > 43
\end{lstlisting}

We can use the language extension \texttt{GeneralizedNewtypeDeriving} to let us derive from a user-defined
typeclass.

Without it, we would have to write
\begin{lstlisting}
class TooMany a where 
  tooMany :: a -> Bool

instance TooMany Int where
  tooMany n = n > 42

newtype Goats = Goats Int deriving (Eq, Show)

-- will do the same thing as the Int instance, but we still have to
-- define it seperately without using the extension.
instance TooMany Goats where
  tooMany (Goats n) = tooMany n
\end{lstlisting}

But with the extension we can write
\begin{lstlisting}
{-# LANGUAGE GeneralizedNewtypeDeriving #-}

class TooMany a where
  tooMany :: a -> Bool

instance TooMany Int where
  tooMany n = n > 42

-- we just deriving from TooMany
newtype Goats =
  Goats Int deriving (Eq, Show, TooMany)
\end{lstlisting}

