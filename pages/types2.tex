\section{More on types}

\subsection{Record types}
Haskell has support for record types which can basically be seen as Haskells version of a C struct.

Let's create a type to represent a \emph{Person}.

\begin{lstlisting}
--                   fName  lName  age gender height
data Person = Person String String Int String Float

firstName :: Person -> String
firstName (Person s _ _ _ _) = s

lastName :: Person -> String
lastName (Person _ s _ _ _) = s

...
\end{lstlisting}

Yikes thats not readable, lets write it in the more readable \emph{record syntax} which is \textbf{identical} to the above syntax but much more organized.

\begin{lstlisting}
data Person = Person { firstName :: String
                     , lastName  :: String
                     , age :: Int
                     , gender :: String
                     , height :: Float
                     } deriving (Show)
\end{lstlisting}

