\section{Functions}
Functions are defined like add x y = x + y. All functions are pure, unless stated otherwise which means
they cannot modify state, or perform \underline{side effects} like input output, writing to a database, etc

\subsection{Currying}
By default, all functions are curried in Haskell. Given the following type signature

\begin{lstlisting}
-- add takes two ints and returns an int
add :: Int -> Int -> Int
\end{lstlisting}

we would say that the function \textbf{add} takes two integers, and returns an integer.
However, because of how currying works it actually means.

\begin{itemize}
    \item \textbf{add} takes a \emph{single} integer parameter
    \item and returns a function that takes another integer parameter, and eventually
    will return an integer itself.
\end{itemize}

Currying is useful because it means we can create new functions by \emph{partially applying}
functions to parameters.

\begin{lstlisting}
-- desugared form of the above add function
add :: Int -> (Int -> Int)

addOne :: Int -> Int
addOne x = add 1

-- addOne 5 desugars to
addOne 5
((add 1) 5)
6
\end{lstlisting}

\subsection{Higher-order functions}
Higher order functions are functions that can accept functions as parameters
\emph{flip} is an example of a higher-order function.

Its type signature is \emph{flip :: (a $\rightarrow$ b $\rightarrow$ c) $\rightarrow$ b $\rightarrow$ a $\rightarrow$ c}
and it can be partially applied like so

\begin{lstlisting}
flipOne = flip 1
partialApply = flipOne 2 -- ((flip 1) 2)
\end{lstlisting}

\subsection{Function composition}
Composing functions is like a \emph{right to left} pipeline. \textbf{f . g} can be read
as \textbf{f} after \textbf{g}.

(f $\circ$ g) x = f (g x)
\begin{enumerate}
    \item first applies \emph{g}
    \item then applies \emph{f} to the result of applying \emph{g}
    \item and makes a new function which takes a parameter \emph{x}
\end{enumerate} 

Example of function composition in action
\begin{lstlisting}
negate . sum $ [1,2,3,4,5]

-- which is equivalent to
negate (sum [1,2,3,4,5])
negate (15)
\end{lstlisting}

\footnote{
    \$ is used since function application has the highest precedence, so Haskell will think we mean negate . 15.
    Alternatively, you can wrap it in brackets like \textbf{negate . (sum [1,2,3,4,5])}}