\section{More operations on lists}

\subsection{Filtering}
\texttt{filter} has the following type signature \emph{filter} :: (a $\rightarrow$ b) $\rightarrow$ [a] $\rightarrow$ [a]. \emph{filter} 

\begin{lstlisting}
filter _ []   = []
filter pred (x:xs)
  | pred x    = x : filter pred xs
  | otherwise = filter pred xs
\end{lstlisting}

Filter all the \textbf{even} numbers from the list, removing the odd ones.
\begin{lstlisting}
filter even [1..10]
-- [2,4,6,8,10]
\end{lstlisting}

Filters can have anonymous lambda syntax instead of directly passing higher-order functions to it.
\begin{lstlisting}
filter (\x -> (x `rem' 2 == 0)) [1..20]
-- [2,4,6,8,10,12,14,16,18,20]
\end{lstlisting}

\subsection{Zipping Lists}
Zipping lists is a means of \textbf{combining} values from multiple lists into a single list. \texttt{zipWith} allows you to use a combining function to product a list of results by \underline{zipping} two lists together.

\begin{lstlisting}
zip [1, 2, 3] [4, 5, 6]
-- [(1,4), (2,5), (3,6)]

-- The lists can be different types
zip [1, 2, 3] ['a', 'b', 'c']
-- [(1,'a'), (2,'b'), (3,'c')]
\end{lstlisting}

We can use \texttt{unzip} to unzip a zipped list and recover the contents of it before it was zipped with \texttt{zip} or \texttt{zipWith}.

\emph{zipWith :: (a $\rightarrow$ b $\rightarrow$ c) $\rightarrow$ [a] $\rightarrow$ [b] $\rightarrow$ [c])}
